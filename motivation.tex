\section{Background and Motivation}

In this section we first motivate the need to detect and mitigation
network-based attacks in the cloud and then discuss the limitation of current 
approaches.

\subsection{Network-Based Attacks in the Cloud.}
The cloud network we study comprises multiple geographically
replicated datacenters connected to each other and to the Internet via
edge and core routers (collectively called border routers).
%
The provider hosts multiple services and each hosted service is
assigned a public virtual IP (VIP) address; we use
the terms VIP and service interchangeably.
%
User requests are typically load balanced across a pool of servers
that are assigned direct IP (DIP) addresses for intra-datacenter
routing. 
%\ramesh{Can a service be hosted on multiple sites? Does it have one VIP?}
%
Incoming traffic first traverses the border routers, then security
appliances like firewalls, DDoS protection appliances and intrusion detection systems, before
hitting the load balancers that distribute traffic across service DIPs.
%
In this paper, we study network-based attacks on cloud infrastructure
(henceforth, \emph{cloud attacks} or simply attacks).

Two aspects of cloud infrastructure make it important to study cloud
attacks.
%
First, compared to enterprise-hosted services, cloud services have
much greater diversity and scale; the cloud provider we study hosts
more than 10,000 services ranging from web storefronts, media streaming,
mobile apps, storage and backup to large online marketplaces. 
%
Unfortunately, this also makes the cloud a %attractive %natural 
target for attackers; a single, well-executed attack can cause more direct
and collateral damage than individual attacks on enterprise-hosted services.
%
While such a large service diversity allows observing 
a wide variety of \emph{inbound} attacks, it 
also makes it hard to distinguish attacks from legitimate traffic 
as these services likely generate {\em all possible} traffic patterns in normal operation.
%
%\ramesh{Navendu, please check and elaborate on this point.}

Second, attackers can abuse the cloud resources to launch \emph{outbound} attacks.
For instance, they can launch brute-force attacks (e.g., password guessing) to 
compromise vulnerable VMs 
%Further, poor security practices and lack of control on third-party tenants may expose
%vulnerabilities which attackers would be likely to compromise and exploit. 
%The VMs running services on the cloud can be compromised using automated, vulnerable VMs 
and gain bot-like control of the infected ones.
%
These compromised VMs may be used for a variety of adversarial purposes such as
YouTube click fraud, BitTorrent hosting, illegal Bitcoin
mining, send SPAM, propagate malware, or launch bandwidth-flooding DoS attacks
(cloud providers typically cap outgoing bandwidth per VM, but not in aggregate across a tenant.).

\subsection{Limitations of current approaches.}
%
While many of these approaches are relevant to cloud defense (such as
end-host filtering, and hypervisor controls), commercial DDoS-protection
 appliances are commonly deployed for dedicated attack detection and mitigation. 
 %
However, from the observation of recent attacks, we argue that those commercial hardware appliances 
are inadequate for attacks in cloud scale, for
three reasons.

% While current commercial security appliances are effective at safeguarding a variety of attacks,
% they have three key limitations for deployment at cloud scale.

\parae{High Cost.} These hardware boxes introduce unfavorable cost
vs. capacity tradeoffs. In particular, they cost hundreds of thousands
up to a few million dollars per box, but can {\em only} handle up to
tens of gigabytes of traffic and risk failure under both network-layer
and application-layer DDoS attacks.
%In a 2011 survey, 36\% and 42\% of the respondents indicated failure of a firewall due to  at the application layer and network layer, respectively
%~\cite{2011 ADC Security Survey Global Findings, http://goo.gl/A3b2Q}.
~\cite{ADC}

Thus, to handle
traffic volume at cloud scale and increasingly high-volume DoS attacks 
(e.g., 300 Gbps+~\cite{Q4report}), this approach would incur significant costs. Further, these devices
need to be deployed in a redundant manner further increasing procurement and 
operational costs. 
%for fault tolerance. 

\parae{Inflexibility.} Since these devices run proprietary software, they limit
how operators can configure them to handle the increasing diversity of attacks today. 
Given lack of rich programmable interfaces, 
operators are forced to specify and manage a large number of policies themselves  
for controlling traffic e.g., set thresholds for different protocols, ports, cluster, VIPs 
at different time granularities. Further, 
they limit effectiveness against increasingly 
sophisticated attacks.
% such as zero-day attacks. %detection and patching.
Finally, these third-party devices may not be kept up to date with OS, firmware and builds which risks reducing their effectiveness against attacks. 
%However, today's hardware appliances can only support a limited number of policies (< 1K) and a limited number of mitigation rules (<50). \minlan{why?}

\parae{Risk collateral damage.} Since many attacks can ramp up
in 10s of seconds to a few minutes, a
relatively high detection latency risks overloading the target VMs as
well as the shared cloud infrastructure e.g., firewalls, load
balancers and core links, which may cause collateral damage to
co-hosted tenants.  Similarly, if we cannot revert fast
enough when the attack subsides, we may mistakenly block legitimate
traffic. Given that many security solutions apply traffic
profiling and smoothing techniques to reduce false positives for
attack detection~\cite{peakflow}, they may not be able to act quickly
enough to avoid collateral damage.

\subsection{Detection/mitigation-as-a-service}
To overcome some of these limitations, we argue that we need to
re-think attack detection in the context of clouds.  To this end, we
describe the architecture of \nimbus, a scale-out VM-based service for
detecting and mitigating to network-based attacks in the cloud.  In contrast
to expensive hardware appliances, our key idea is to leverage the core
principles in cloud computing (elastic scaling of resources on demand)
and software-defined networks (programmability of multiple network
layers) to introduce a new paradigm of {\em detection-as-a-service}
and {\em mitigation-as-a-service}.  We first present the requirements
based on our measurement study and then describe its design and
implementation.

\parab{1.} Scaling to match datacenter traffic capacity at the order of hundreds of gigabits per second. 
The service should auto-scale to enable agility and cost-effectiveness.  

\parab{2.} Programmability to handle new and diverse types of network-based attacks, and flexibility to    
allow tenants/operators to configure policies %and roll out updates %in an attack-specific and traffic-based patterns. %schemes that are 
specific to the traffic patterns and attack characteristics.

\parab{3.} Fast and accurate detection and mitigation for attacks that ramp-up within one minute; 
once the attack subsides, we should revert the mitigation to avoid blocking legitimate traffic. 

\parab{4.} Robustness to prevent sophisticated attacks from a wise adversary, who knows our system well and want to victimize either the detection service or individual cloud service. 



